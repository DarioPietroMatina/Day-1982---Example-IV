One of the limitations of the Solow model is that growth is not a smooth and continuous process. As a matter of fact, recessions and irregular growth cycles are at the core of the mechanism of economic development. Take for instance any developed country's time series on national GDP and it e \\In order to account for such irregularities, Day (1982) takes the theoretical framework developed by Solow and extends it by incorporating into the model insights from the so-called Chaos Theory. As a consequence, oscillating behaviours and chaotic patterns arise.
\\
\\
For the sake of simplicity, assume there is no technological progress ($g=0$).
Suppose that the saving rate is no longer exogenous but depends positively on the interest rate such that saving function $s(k)$ becomes:
$$s(k_t)=q\Big{(}1-\frac{c}{r_t}\Big{)}\frac{k_t}{y_t}$$
where $q$ and $c$ are exogenous parameters whereas $r_t$ is the interest rate at time $t$ that, under the assumption of competitive markets and Cobb-Douglas production function, is given by 
$$r_t=\alpha \frac{y_t}{k_t}$$.
\\
Furthermore, assume that the law of motion of capital takes the following form: 
$$k_{t+1}=\frac{min\{ (1+\rho) k_{t}, \ s(k_t)y_t \} }{1+n}$$
\\
where $\rho$ is the maximal growth rate of capital. Day (1957) proposes several explanations why such a limitation may arise, including behavioural reasons or possible costs of adjustment.
